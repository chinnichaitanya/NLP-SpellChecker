\documentclass{article}

\usepackage{amsmath}
\usepackage{graphicx}
\usepackage{caption}
\usepackage{subcaption}
\usepackage{color}
\usepackage{listings}
\usepackage{booktabs}
\usepackage{forest}

\title{
	\underline{CS6370: Natural Language Processing}\\
	\textbf{Spell Check Assignment Report}\\
	Indian Institute of Technology Madras
}
\author{
	\textbf{Chinni Chaitanya, EE13B072}\\
	\texttt{ee13b072@smail.iitm.ac.in}	
	\and
	\textbf{Venkatesh Maligireddy, EE13B041}\\
	\texttt{ee13b041@smail.iitm.ac.in}
	\and
	\textbf{Swaroop Kotni, EE13B030}\\
	\texttt{ee13b030@smail.iitm.ac.in}
	\and
	\textbf{Pronnoy Noel, EE13B029}\\
	\texttt{ee13b029@smail.iitm.ac.in}
}

\begin{document}
	\pagenumbering{gobble}
	\maketitle
	\newpage
	
	\pagenumbering{roman}
	\tableofcontents
	\newpage
	
	\pagenumbering{arabic}
	\section{Introduction}
		This assignment attempts to implement \textit{spell checker and correction} programs for text, which will correct the erroneous word\footnote{In phrases and sentences, the erroneous word need not be misspelled but might be contextually incorrect.} present in it. The assignment is categorized into the following three parts,
		\begin{description}
			\item[$\bullet$] \textbf{Word checker}, a program which checks and corrects standalone erroneous words\footnote{The word given might be correct word also.}.
			\item[$\bullet$] \textbf{Phrase checker}, a program which examines phrases and corrects the erroneous words. We assume \textit{phrases} contain about 5 words.
			\item[$\bullet$] \textbf{Sentence checker}, a program which examines sentences and corrects the erroneous words. We assume \textit{sentences} contain about 30 words.
		\end{description}
		In all the three cases, we assume that there is only \textbf{one} erroneous word.
		
	\section{Data generation}
		\subsection{Corpa used for data generation}
			
		\subsection{Data generated using these corpa}
			
	\section{Word checker}
		....... Add content .......
		
		\subsection{Generation of candidates}		
		
		\subsection{Ranking the candidates using Bayesian probability}
		
		\subsection{Ranking the candidates Phonetically}
		
		\subsection{Combined ranking of candidates, using both Bayesian and Phonetic ranking}		
		
		\subsection{Observations}		
		
		
	\section{Phrase and checker}
				
		
	
\end{document}